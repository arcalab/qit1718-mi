%\documentclass[trans]{beamer}
\documentclass{beamer}
\usetheme{Singapore}
\setbeamercovered{transparent} 
\usepackage{amsmath}
\usepackage{stmaryrd}
\usepackage{src/prooftree}

\usepackage[all]{xy}
\def\igual{=}
\def\caixa#1{\medskip
  \begin{center}
  \fbox{\begin{minipage}{0.9\textwidth}\protect{#1}\end{minipage}}
  \end{center}}
  
\def\hdd{\fun{hd}}
\def\tll{\fun{tl}}
\def\swap{\fuc{swap}}
\def\twist{\fuc{twist}}
  %------ using color ---------------------------------------------------------
\usepackage{color}
%input{rgb}
\def\black{\color{black}}
\def\blue{\color{blue}}
\def\red{\color{red}}

\def\true{\fuc{true}}
\def\false{\fuc{false}}
\def\dimp{\mathbin{\equiv}}
\def\imp{\mathbin{\Rightarrow}}
\def\rimp{\mathbin{\Leftarrow}}
\def\limp{\mathbin{\multimap}}
\newcommand{\sse}{\mathbin{\, \operatorname{\textsf{iff}} \,}}
\def\e{\mathbin{\wedge}}
\def\ou{\mathbin{\vee}}

  
  \input{src/newmacros.mac}
  
  
\title{
	Mathematics for Computer Science
}
\author{ Jos\'{e} Proen\c{c}a \& Alexandre Madeira\\ (slides from Luis Soares Barbosa)\\ ~}

\institute{
 \includegraphics[height=1.5cm]{img/HASLab-logo} \hspace{1cm}
 \includegraphics[height=1.5cm]{img/um} \hspace{1cm} 
 % \includegraphics[height=1.5cm]{UNU-EGOV}
 }
%\institute{HASLab INESC TEC \& Universidade do Minho \& UNU-EGOV - United Nations University \\
%Braga, Portugal}
\date{
\begin{tabular}{c}
\\
	\textbf{Expressions} \\  September-October,  2017

\\
\end{tabular}
}

\begin{document}

\frame[plain]{\titlepage}


\begin{slide}{Expressions}
\begin{itemize}
\item \dkb{Syntax} refers to the structure of expressions, or the rules for putting symbols together to form an expression. 
\item \dkb{Semantics} refers to the meaning of expressions, or how they are evaluated.
\end{itemize}
\end{slide} 

\begin{slide}{Expressions}
Recall the syntax of conventional mathematical expressions:

\[
E\; \::=\; c \mid v \mid (E) \mid \circ E \mid E * E
\]

\noindent where $c$ belongs to a set of constants, $v$ to a set of variables, $\circ$ is a unary operator and $*$ a binary operator.

\end{slide} 

\begin{slide}{Textual substitution}

$$E[x := E']$$
~\\

introduced as an \dkb{inference rule}:

\begin{center}
\begin{prooftree}
E
\justifies
E[x := E']
\thickness=0.08em
\using
{\text{(substitution)}}
\end{prooftree}
\end{center}
\end{slide} 

\begin{slide}{Equality}
\begin{itemize}
\item \dkb{Semantic} characterization ---  in terms of expression evaluation: Evaluation of the expression X = Y in a state yields the value true if expressions X and Y have the same value and yields false if they have different values. \\
\item \dkb{Syntactic} characterization ---  in terms of  a set of laws that can be used to show that two expressions are equal, without calculating their values. 
\end{itemize}

\end{slide} 

\begin{slide}{Equality}
\begin{description}
\item[reflexivity:] $X = X$ \vspace{5mm}
\item[symmetry:] $(X = Y) = (Y = X)$ \vspace{5mm}
\item[transitivity:] 

%\begin{center}
\begin{prooftree}
X = Y , Y= Z
\justifies
X = Z
\thickness=0.08em
%\using
%{\text{(substitution)}}
\end{prooftree}
%\end{center}
\vspace{5mm}

\item[Leibniz:] 

%\begin{center}
\begin{prooftree}
X = Y 
\justifies
E[z := X] = E[z := Y] 
\thickness=0.08em
%\using
%{\text{(substitution)}}
\end{prooftree}
%\end{center}


\end{description}
\end{slide} 


\begin{slide}{Equality}
\begin{block}{Functions}
\begin{itemize}
\item A function is a rule for computing a value from another value 
\item \dkb{application} (can be defined in terms of textual substitution)
% $$f (x) = f\, x = E[z := X]$$
% for $f = E$
$$ f(v) = E $$
then
$$f~X = E[v:=X]$$
\item The Leibniz rule links equality and function application:
~\\

\begin{prooftree}
X = Y 
\justifies
f\, X = f\, Y
\thickness=0.08em
\using
{\text{(Leibniz)}}
\end{prooftree}
~\\

\item \dkb{composition}  %(types)
\end{itemize}
\end{block}
\end{slide} 


\begin{slide}{Boolean expressions}

\fbox{i.e. expressions taking values in $\enset{\true, \false}$}

\caixa{Boole's great contribution was an algebraic basis for logic, something Leibniz had dreamed about 200 years earlier and that De Morgan was unable to devise. \\
In \emph{The Laws of Thought} Boole's aim was to \emph{\dkb{"investigate the fundamental laws ... by which reasoning is performed, ... give expression to them in the language of a Calculus, and upon this foundation ... establish the Science of Logic ... "}}.
}
 \end{slide} 
 
 \begin{slide}{Boolean expressions}

\begin{itemize}
\item \dkb{Syntax}: Boolean connectives
$$ 
E\; \::=\; \true \mid \false \mid v \mid (E) \mid \neg E \mid E \land E \mid E \lor E \mid E \implies E
$$

\item \dkb{Semantics}: interpretation; truth tables
\item \dkb{Equality versus equivalence}
\begin{itemize}
\item $=$ has a \dkb{conjunctional} flavour: \\ $b = c = d$ is an abbreviation for  $b \dimp c \wedge c \dimp d$
\\ Therefore $b = c = d$ and $(b = c) = d$ are different.
\item $\dimp$ is \dkb{associative} (thus not conjunctional)
\end{itemize}
\end{itemize}

\end{slide} 


 \begin{slide}{Boolean expressions}

\begin{eqnarray*}
& & \false = \false = \true
%
\just \igual{ $=$ is abbreviates a conjunction }
%
(\false = \false) \e (\false = \true)
%
\just \igual{ evaluate both $=$ }
%
\true \e \false
%
\just \igual{ evaluation $\e$ }
%
\false
%
\end{eqnarray*}

\end{slide} 


\begin{slide}{Boolean expressions}

\begin{eqnarray*}
& & \false \dimp \false \dimp \true
%
\just \igual{ evaluates first $\dimp$ }
%
\true \dimp \true
%
\just \igual{ evaluate $\dimp$ }
%
\true \end{eqnarray*}

Treat the conjunctional use of $=$ and other operators as \dkb{syntactic sugar}, i.e. as an extension to the basic definition of expressions to make writing some expressions easier. Whenever an expression that contains this syntactic sugar is to be evaluated or manipulated, first remove the syntactic sugar.
\end{slide} 


 \begin{slide}{Boolean expressions: Satisfiability and validity}

\begin{itemize}
\item A Boolean expression  is \dkb{satisfied} in a state if its value is true in that state; 
\item it  is \dkb{satisfiable} if there is a state in which it is satisfied;
\item it  is \dkb{valid} if it is satisfied in every state;
\item  A valid boolean expression is called a \dkb{tautology}
\end{itemize}

\end{slide} 

\begin{slide}{Boolean expressions: Duality}

The dual $E_D$ of a boolean expression $E$ is built from $E$ by interchanging occurrences of $\true$ and $\false$, $\e$ and $\ou$, $\equiv$ and $\not\equiv$, $\imp$ and $\not \rimp$, and $\rimp$ and $\not\imp$,

\begin{block}{Duality meta-theorem}

\begin{itemize}
\item $E$ is valid iff $\neg E_D$ is valid
\item $E \dimp F$ is valid iff $E_D\dimp F_D$ is valid
\end{itemize}
\end{block}
\end{slide} 


\begin{slide}{Modelling natural language propositions}
\begin{block}{Proposition}
is a statement that can be interpreted as being either true or false
\end{block}
\begin{block}{Modelling natural language requirements as Boolean expressions}
\begin{itemize}
\item Introduce boolean variables to denote subpropositions.
\item Replace these subpropositions by their corresponding boolean variables.
\item Translate the results of step 2 into a Boolean expression,
\end{itemize}
\end{block}
\end{slide}


\begin{slide}{Modelling natural language propositions}


\begin{block}{Exercises}

If Superman was able and willing to prevent evil, he would do so. If Superman was unable to prevent evil, he would be impotent; if he was unwilling to prevent evil, he would be malevolent. Superman does not prevent evil. If Superman exists, he is neither impotent nor malevolent. Therefore, Superman does not exist.
\end{block}
\end{slide} 

%
% AP & WP -> P
% �AP -> Imp
% �WP -> Mal
% �P
% E -> �(Imp Or Mal)
% ---------
% �E


\end{document}


\section{Models}

%\begin{slide}{Why do we need formal models?}
% \begin{center}
% �\includegraphics[scale=0.95]{c1}
% \end{center}
% \end{slide} 
%
%\begin{slide}{Why do we need formal models?}
%Notice the "magic words"\\
%\begin{itemize}
%\item much \textbf{deeper understanding} of the systems we invent
%\item greater ability to make \textbf{predictions}
%\item about the behavior of  \textbf{complex}, connected, and interacting systems
%\item better  \textbf{tools}
%\end{itemize}
%\end{slide} 


\begin{slide}{... on second thoughts}
IT became ubiquitous to modern life long before a \textbf{solid scientific methodology},
 let alone formal foundations, has been put forward\\~\\
 
 \pause
Formal modelling aims at lifting to Informatics the  \dkb{\textbf{problem-solving strategy}} one got used to from
school physics:
\vspace{0.1cm}
\begin{itemize}
\item \dgold{understand} the problem
\item \dgold{build} a mathematical \dkb{model} of it
\item \dgold{animate}  and \dgold{reason} in such a model
\item \dgold{upgrade} the model whenever necessary
\item \dgold{calculate} a \dkb{solution} and \dgold{implement} it
\end{itemize}
\end{slide}


\begin{slide}{Models}
\begin{itemize}
\item introduce \textbf{rigour} (in which other engineering would a warranty be replaced by a disclaimer?)
\item develop \textbf{abstraction} skills
\item help developers to \textbf{manage complexity} and assess alternative designs in early stages
\end{itemize}
\begin{flushright}
 \dkb{\emph{A good model guides your thinking, a bad one wraps it}\\ Brian Marick}
\end{flushright}
\end{slide}


\begin{slide}{Models}
\begin{block}{Example}
Mary has twice as many apples as John. Mary throws half her apples away, because they are rotten, and John eats one of his. Mary still has twice as many apples as John. How many apples did Mary and John have initially?
\end{block}
\end{slide}

\begin{slide}{Models}
\begin{block}{Example}
\begin{equation*}
\left\{
  \begin{array}{l}
    m  =   2 \comp j \\
    m/2  =  2 \comp (j-1) \\
  \end{array}
\right.
\end{equation*}
\end{block}
\end{slide}


\begin{slide}{Models}
\begin{itemize}
\item 
A mathematical model may be more understandable, concise, precise, or rigorous than an informal description written in a natural language.
\item
Answers to questions about an object or phenomenon can often be computed directly using a mathematical model of the object or phenomenon.
\item
Mathematics provides methods for reasoning: for manipulating expressions, for proving properties from and about expressions, and for obtaining new results from known ones. This reasoning can be done without knowing or caring what the symbols being manipulated mean.
\end{itemize}
\end{slide}


\section{Rigorous reasoning}

\begin{slide}{Reasoning}

Aims at
\begin{itemize}
\item calculating solutions (implementations)
\item producing evidence (proof, i.e. precise, honest explanations)
\end{itemize}

\begin{center}
\fbox{Logic is the glue that binds together methods of reasoning }
\end{center}
\end{slide}

\begin{slide}{Reasoning}
\begin{itemize}
\item
IT-driven societies  require from people a higher degree of \dkb{mathematical
literacy}, i.e.
\begin{itemize}
\item the ability to reason in terms of abstract models 
\item and effectively use of logical arguments and mathematical calculation
\item ... in daily 
business practice
\end{itemize}
\item  High-valued programmers are heavy users of \dgold{logic}:
%which entails the need for \dkb{earlier} introduction and \bblue{explicit} use of logic in middle and high school
%~\\
%\item
 but a heavy use of logic also requires more \dgold{concise} ways of expression and
notations amenable to formal, \dgold{systematic manipulation}
\end{itemize}
\end{slide}





\begin{slide}{Proofs}
\small

{\bf Ubiquitous }
\begin{itemize}
\item certification of critical systems 
\item e-commerce security
\item automatic contracts in global infra-structures (cf, \dgold{proof-carrying code})
\end{itemize}
\vspace{0.5cm}

{\bf However }
\begin{itemize}
\item  proofs are usually omitted from schoolroom practice
\item when present, they are non-systematic, informal, and seen as difficult
\end{itemize}

\end{slide}

\begin{slide}{What we see}
\begin{itemize}
\item 
IT-driven societies  require from people a higher degree of \dkb{mathematical
literacy}, i.e.
\begin{itemize}
\item the ability to reason in terms of abstract models 
\item and effectively use of logical arguments and mathematical calculation
\item ... in daily 
business practice
\end{itemize}
\item  High-valued programmers are heavy users of \dgold{logic}:
%which entails the need for \dkb{earlier} introduction and \bblue{explicit} use of logic in middle and high school
%~\\
%\item
 but a heavy use of logic also requires more \dgold{concise} ways of expression and
notations amenable to formal, \dgold{systematic manipulation}
\end{itemize}
\end{slide}



\begin{slide}{Calculational proofs}

Example: \dgold{is $\log(2) + \log(7)\; <\; \log(3) + \log(5)$?}
\vspace{0.5cm}

The \dkb{classical} proof:   \dkb{verify the hypothesis} 
% $\log(2) + \log(7)\; <\; \log(3) + \log(5)$ and then verifies 


\begin{quote}
\begin{tabular}{llr}
(1) \hspace{0.2cm} & \text{ function $\log$ is strictly increasing} \\
(2) \hspace{0.2cm} & $\log(x \times y) = \log(x) + \log(y)$ \\
(3) \hspace{0.2cm} & $14 < 15 $ \\
(4) \hspace{0.2cm} & $14 = 2 \times 7$ \text{and} $15 = 3 \times 5$\\
(5) \hspace{0.2cm} & $\log(14) < \log(15)$ &  \text{by (1), (3)}\\
(6) \hspace{0.2cm} & $\log(2) + \log(7) < \log(3) + \log(5)$ &  \text{by (2), (4), (5)} \\
\end{tabular}
\end{quote}
\end{slide}

\begin{slide}{Calculational proofs}

The \dgold{classical} proof:
\begin{itemize}
\item easy to follow
\item  but  provides poor  intuition on the problem 
\item hard memorize or reproduce
\item most probably it was not made, originally, by the
order in which it is presented
\item  fails to attract students enthusiasm
\end{itemize}
\end{slide}

\begin{slide}{Calculational proofs}


The \dgold{calculational} proof:
\begin{eqnarray*}
& & \log(2) + \log(7) \; \; \square \; \;  \log(3) + \log(5)
%
\just \igual{ function  $\log$ distributes over multiplication }
%
\log(2 \times 7)  \; \; \square \;  \; \log(3 \times 5)
%
\just \igual{ routine arithmetic }
%
\log(14)   \; \; \square \; \;  \log(15)
%
\just \igual{ $14 < 15$ and function $\log$ is strictly increasing }
%
\; \square \: \; \; \;  \text{is}\: \; \; \; \;  <
%
\end{eqnarray*}
\end{slide}


\begin{slide}{Calculational proofs}

The \dgold{calculational} proof:
\begin{itemize}
\item The starting point is not an hypothesis to verify,  but the problem itself. 
\item The proof starts by identifying an unknown $\square$
 which stands for an order relation 
 \item and 
 proceeds by the identification and application of whatever known properties are
 useful in its determination
 \item  being essentially syntax driven it builds intuition and meaning.
\end{itemize}
\end{slide}





\end{document}

I