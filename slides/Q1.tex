
\documentclass{beamer}

\usetheme{Singapore}
\setbeamercovered{transparent} 
\usepackage{amsmath}
\usepackage{stmaryrd}

\usepackage[all]{xy}

\def\igual{=}
\def\caixa#1{\medskip
  \begin{center}
  \fbox{\begin{minipage}{0.9\textwidth}\protect{#1}\end{minipage}}
  \end{center}}
  
\def\hdd{\fun{hd}}
\def\tll{\fun{tl}}
\def\swap{\fuc{swap}}
\def\twist{\fuc{twist}}
  %------ using color ---------------------------------------------------------
\usepackage{color}
\def\black{\color{black}}
\def\blue{\color{blue}}
\def\red{\color{red}}

  
 \input{src/newmacros.mac}
  
\title{
	Mathematics for Computer Science
}
\author{ Jos\'{e} Proen\c{c}a \& Alexandre Madeira\\ (slides from Luis Soares Barbosa)\\ ~}

\institute{
 \includegraphics[height=1.5cm]{img/HASLab-logo} \hspace{1cm}
 \includegraphics[height=1.5cm]{img/um} \hspace{1cm} 
 % \includegraphics[height=1.5cm]{UNU-EGOV}
 }
%\institute{HASLab INESC TEC \& Universidade do Minho \& UNU-EGOV - United Nations University \\
%Braga, Portugal}
\date{
\begin{tabular}{c}
\\
	\textbf{Modelling and Calculi} \\ September-October,  2017
\\
\end{tabular}
}

\begin{document}

\frame[plain]{\titlepage}

\section{Introduction}
\begin{slide}{Why Maths for Software Engineering?}
The phrase \textbf{software engineering} dates back to the Garmisch NATO conference in 1968:

\begin{center}
\begin{minipage}{8cm}
 \emph{\blue The phrase ``software engineering'' was deliberately chosen as being  \textbf{provocative}, in implying the need for software manufacture to be based on the types of \textbf{theoretical foundations} and practical disciplines, that are \textbf{traditional in the established branches of engineering}.}\\
\begin{flushright}
(Naur and Randell, 1969)
\end{flushright}
\end{minipage}
\end{center}

Provocative or not, 45 years later,\\
 \textbf{how 'scientific' do such foundations turn out to be?}

\end{slide} 


\begin{slide}{Science vs Engineering}

from \emph{How Science Was Born in 300BC and Why It Had to Be Reborn}:

\begin{center}
\begin{minipage}{8cm}
 \emph{\blue The immense usefulness of exact science consists in providing \textbf{models} of the real world within which there is a guaranteed method for telling false statements from true. (...) Such models, of course, allow one to  \textbf{describe and predict} natural phenomena, by translating them to the theoretical level via \textbf{correspondence rules}, then solving the ``\textbf{exercises}'' thus obtained and translating the solutions obtained back to the real world.}\\
\begin{flushright}
(Lucio Russo, 2007)
\end{flushright}
\end{minipage}
\end{center}

\end{slide} 


\begin{slide}{Science vs Engineering}
\begin{itemize}
\item \emph{\dgold{...  providing \textbf{models} of the real world within which there is a guaranteed method for telling false statements from true ...}} 
\item \emph{\dgold{...   solving the ``\textbf{exercises}'' thus obtained ...}} 
\end{itemize}
~\\
In other words:
\begin{itemize}
\item abstract and precise \dkb{\textbf{models}} ...
\item ... and calculi for \dkb{\textbf{reasoning about  and within} such models}
\end{itemize}
i.e.,
\begin{flushright}
\fbox{the hallmark of any \textbf{true Engineering discipline}}
\end{flushright}
\end{slide} 




\section{Models}



\begin{slide}{... on second thoughts}
IT became ubiquitous to modern life long before a \textbf{solid scientific methodology},
 let alone formal foundations, has been put forward\\~\\
 
 \pause
\textbf{Formal modelling} aims at lifting to Informatics the  \dkb{\textbf{problem-solving strategy}} one got used to from
school physics:
\vspace{0.1cm}
\begin{itemize}
\item \dgold{understand} the problem
\item \dgold{build} a mathematical \dkb{model} of it
\item \dgold{animate}  and \dgold{reason} in such a model
\item \dgold{upgrade} the model whenever necessary
\item \dgold{calculate} a \dkb{solution} and \dgold{implement} it
\end{itemize}
\end{slide}


\begin{slide}{Models}
\begin{itemize}
\item introduce \textbf{rigour} (in which other engineering would a warranty be replaced by a disclaimer?)
\item develop \textbf{abstraction} skills
\item help developers to \textbf{manage complexity} and assess alternative designs in early stages
\end{itemize}
\begin{flushright}
 \dkb{\emph{A good model guides your thinking, a bad one wraps it}\\ Brian Marick}
\end{flushright}
\end{slide}


\begin{slide}{Models}
\begin{block}{Example}
Mary has twice as many apples as John. Mary throws half her apples away, because they are rotten, and John eats one of his. Mary still has twice as many apples as John. How many apples did Mary and John have initially?
\end{block}
\end{slide}

\begin{slide}{Models}
\begin{block}{Example}
\begin{equation*}
\left\{
  \begin{array}{l}
    m  =   2 \comp j \\
    m/2  =  2 \comp (j-1) \\
  \end{array}
\right.
\end{equation*}
\end{block}
\end{slide}


\begin{slide}{Models}
\begin{itemize}
\item 
A mathematical model may be \dkb{more understandable, concise, precise}, or rigorous than an informal description written in a natural language.
\item
\dkb{Answers to questions} about an object or phenomenon can often be computed directly using a mathematical model of the object or phenomenon.
\item
Mathematics provides \dkb{methods for reasoning}: for manipulating expressions, for proving properties from and about expressions, and for obtaining new results from known ones. This reasoning can be done \dkb{without knowing or caring what the symbols being manipulated mean}.
\end{itemize}
\end{slide}


\section{Rigorous reasoning}

\begin{slide}{Reasoning}

Aims at
\begin{itemize}
\item calculating solutions (implementations)
\item producing evidence (proof, i.e. precise, honest explanations)
\end{itemize}

\begin{center}
\fbox{Logic is the glue that binds together methods of reasoning }
\end{center}
\end{slide}

\begin{slide}{Reasoning}
\begin{itemize}
\item
IT-driven societies  require from people a higher degree of \dkb{mathematical
literacy}, i.e.
\begin{itemize}
\item the ability to reason in terms of abstract models 
\item and effectively use of logical arguments and mathematical calculation
\item ... in daily 
business practice
\end{itemize}
\item  High-valued programmers are heavy users of \dgold{logic}:
%which entails the need for \dkb{earlier} introduction and \bblue{explicit} use of logic in middle and high school
%~\\
%\item
 but a heavy use of logic also requires more \dgold{concise} ways of expression and
notations amenable to formal, \dgold{systematic manipulation}
\end{itemize}
\end{slide}





\begin{slide}{Proofs}
\small

{\bf Ubiquitous }
\begin{itemize}
\item certification of critical systems 
\item e-commerce security
\item automatic contracts in global infra-structures (cf, \dgold{proof-carrying code})
\end{itemize}
\vspace{0.5cm}

{\bf However }
\begin{itemize}
\item  proofs are usually omitted from schoolroom practice
\item when present, they are non-systematic, informal, and seen as difficult
\end{itemize}

\end{slide}

% \begin{slide}{What we see}
% \begin{itemize}
% \item 
% IT-driven societies  require from people a higher degree of \dkb{mathematical
% literacy}, i.e.
% \begin{itemize}
% \item the ability to reason in terms of abstract models 
% \item and effectively use of logical arguments and mathematical calculation
% \item ... in daily 
% business practice
% \end{itemize}
% \item  High-valued programmers are heavy users of \dgold{logic}:
% %which entails the need for \dkb{earlier} introduction and \bblue{explicit} use of logic in middle and high school
% %~\\
% %\item
%  but a heavy use of logic also requires more \dgold{concise} ways of expression and
% notations amenable to formal, \dgold{systematic manipulation}
% \end{itemize}
% \end{slide}



\begin{slide}{Calculational proofs}

Example: \dgold{is $\log(2) + \log(7)\; <\; \log(3) + \log(5)$?}
\vspace{0.5cm}

The \dkb{classical} proof:   \dkb{verify the hypothesis} 
% $\log(2) + \log(7)\; <\; \log(3) + \log(5)$ and then verifies 


\begin{quote}
\begin{tabular}{llr}
(1) \hspace{0.2cm} & \text{ function $\log$ is strictly increasing} \\
(2) \hspace{0.2cm} & $\log(x \times y) = \log(x) + \log(y)$ \\
(3) \hspace{0.2cm} & $14 < 15 $ \\
(4) \hspace{0.2cm} & $14 = 2 \times 7$ \text{and} $15 = 3 \times 5$\\
(5) \hspace{0.2cm} & $\log(14) < \log(15)$ &  \text{by (1), (3)}\\
(6) \hspace{0.2cm} & $\log(2) + \log(7) < \log(3) + \log(5)$ &  \text{by (2), (4), (5)} \\
\end{tabular}
\end{quote}
\end{slide}

\begin{slide}{Calculational proofs}

The \dgold{classical} proof:
\begin{itemize}
\item easy to follow
\item  but  provides poor  intuition on the problem 
\item hard memorize or reproduce
\item most probably it was not made, originally, by the
order in which it is presented
%\item  fails to attract students enthusiasm
\end{itemize}
\end{slide}

\begin{slide}{Calculational proofs}


The \dgold{calculational} proof:
\begin{eqnarray*}
& & \log(2) + \log(7) \; \; \square \; \;  \log(3) + \log(5)
%
\just \igual{ function  $\log$ distributes over multiplication }
%
\log(2 \times 7)  \; \; \square \;  \; \log(3 \times 5)
%
\just \igual{ routine arithmetic }
%
\log(14)   \; \; \square \; \;  \log(15)
%
\just \igual{ $14 < 15$ and function $\log$ is strictly increasing }
%
\; \square \: \; \; \;  \text{is}\: \; \; \; \;  <
%
\end{eqnarray*}
\end{slide}


\begin{slide}{Calculational proofs}

The \dgold{calculational} proof:
\begin{itemize}
\item The starting point is not an hypothesis to verify,  but the problem itself. 
\item The proof starts by identifying an unknown $\square$
 which stands for an order relation 
 \item and 
 proceeds by the identification and application of whatever known properties are
 useful in its determination
 \item  being essentially syntax driven it builds intuition and meaning.
\end{itemize}
\end{slide}



\section{This course}

\begin{slide}{Syllabus}
\begin{itemize}
\item Logic: propositional and first order
\item Set theory
\item Functions and relations
\item Partial orders, monoids, trees and graphs
\item Recurrence relations and induction
% \item Introduction to combinatorics
\end{itemize}
\end{slide}


\begin{slide}{Pragmatics}
  \myblock{18 Sep -- 28 Sep (4 weeks)}
  \begin{itemize}
    % \item 18 Sep -- 28 Sep (4 weeks)
    \item 18 Sep -- 28 Sep (2 weeks), Jos\'{e} Proen\c{c}a
    \item 2 Oct -- 12 Oct (2 weeks), Alexandre Madeira
  \end{itemize}
  ~\\[5mm]
  \myblock{Test: 10 and 12 October}
  \myblock{Exam: 13 to 20 December}
\end{slide}

\begin{slide}{References}
\begin{itemize}
\item \dkb{A logical approach to discrete mathematics}, David Gries and Fred Schneider, Springer, 1993.
\item \dkb{Discrete mathematics}, Kevin Ferland, Houghton Mifflin Company, 2009.
\item ... plus \dkb{slides} and \dkb{exercises} ...
\end{itemize}

~\\[5mm]

Other references
\begin{itemize}
  \item \dkb{Naive Set Theory}, Paul Halmos, 1960
  % \item 
\end{itemize}

~\\[3mm]

\myblock{\url{http://mi1718.proenca.org}}

\end{slide}




\end{document}
